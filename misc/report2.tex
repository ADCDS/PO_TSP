\documentclass{article}
\usepackage[utf8]{inputenc}

\title{Pesquisa Operacional - Relatório 10}
\author{Adriel Cardoso dos Santos}
\date{Maio 2018}
\usepackage{amsmath}
\usepackage{amssymb}
\usepackage[]{algorithm2e}
\usepackage{graphicx}
\usepackage[portuges]{babel}


\begin{document}

    \maketitle

    \section{Descrição}

    Esse relatório tem como objetivo comparar os dois modelos que solucionam o problema do cacheiro viajante, um utiliza LazyConstraintCallback
    para adicionar as restrições de subciclos (Modelo LCC), o outro não utiliza o LazyConstraintCallback (Modelo s/ LCC), e adiciona as restrições iterativamente.

    \section{Resultados}

    Foi possível solucionar na otimalidade as instâncias de 10, 25, 50, 75 com tempo viável, as demais demoraram muito.

    \subsubsection{Tempo de execução}
    Essa tabela apresenta o tempo de execução total de cada script para cada instância em milissegundos.
    \begin{center}
        \begin{tabular}{ | l | l | l | l | l | p{5cm} |}
            \hline
            & 10.csv & 25.csv & 50.csv & 75.csv \\ \hline
            Modelo s/ LCC & 961 ms & 2037 ms & 5975 ms & 31503 ms\\
            Modelo LCC & 897 ms & 4503 ms & 120754 ms & 229189 ms\\
            \hline
        \end{tabular}
    \end{center}

    \subsubsection{Ticks}

    \begin{center}
        \begin{tabular}{ | l | l | l | l | l | p{5cm} |}
            \hline
            & 10.csv & 25.csv & 50.csv & 75.csv \\ \hline
            Modelo s/ LCC & 2.64 & 48.98 & 166.93 & 778.82\\
            Modelo LCC & 8.42 & 203.38 & 58388.43 & 75379.51\\
            \hline
        \end{tabular}
    \end{center}


    \section{Conclusão}
    Foi possível observar que o modelo que utiliza o LazyConstraintCallback teve um desempenho muito pior que o modelo iterativo,
    mesmo levando em consideração que o número de ticks para o modelo sem LCC se refere apenas a ultima iteração em que o método $solve()$
    foi chamado, a tabela de tempo de execução nos mostra que o modelo sem LCC foi extremamente mais rápido.
    Acredito que o motivo disso se deve ao fato da implementação do LazyConstraintCallback na API Python do CPLEX ser limitada,
    nessa implementação, toda vez que o callback é chamado, não posso simplesmente adicionar uma restrição utilizando uma expressão
    linear, como é feito no modelo iterativo, preciso tratar os dados, encapsulá-los numa estrutura chamada SparsePair com os nomes
    das variáveis (string) e os respetivos coeficientes, depois preciso adicionar a restrição, chamando a função LazyConstraintCallback.add()
    essa função recebe o SparsePair, uma string que representa um operador de igualdade/desigualdade ("G" indica "$>$", "LE" indica "$<=$", etc.)
    e um float que representa a parte que está a direta da restrição.
    Acredito que toda essa operação de adicionar restrições utilizando nome das variáveis em string e lista de coeficientes adiciona um overhead
    no callback, fazendo com que todo o tempo de execução do script fique mais lento.
    É provável também que eu tenha cometido um erro na implementação, vale ressaltar que a documentação da API Python do CPLEX é extramente confusa,
    com pouca informação sobre as funções/parâmetros de funções, e, praticamente sem exemplos práticos.


\end{document}